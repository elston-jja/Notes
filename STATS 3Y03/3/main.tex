\documentclass[12pt, titlepage, oneside]{article}

\usepackage[margin=0.5in]{geometry}

\usepackage{siunitx, booktabs, amsmath, enumitem, pdfpages,mathrsfs,tabularx,caption, graphicx, pgfplots, textcomp,wrapfig, commath, svg}

\usepackage{parskip}

\usepackage[siunitx]{circuitikz}
\sisetup{detect-weight=true, detect-family=true}

\setlength\parindent{0pt}

\let\oldhat\hat
\let\oldvec\vec
\newcommand{\cross}{\bm{\times}}
\renewcommand{\hat}[1]{\oldhat{\mathbf{#1}}}

\usepackage{bm}
\renewcommand{\vec}[1]{\oldvec{\bm{#1}}}
\renewcommand{\hat}[1]{\oldhat{\bm{#1}}}
\renewcommand{\b}[1]{\textbf{#1}}

\newcommand{\de}[1]{\noindent\fbox{\parbox{\textwidth}{#1}}}

\newcommand{\be}{\begin{equation*}}
\newcommand{\ee}{\end{equation*}}

\begin{document}

\setcounter{section}{2}
\setcounter{page}{6}

\section{Probability}

Probability is a function assigned to the events of a sample space and represent our belief in what the outcomes of an experiment are likely to be.

Suppose our sample space $S$ has finite points \{$s_1,s_2,..,s_N$\}. We assign non-negative numbers $p_i$ (probability) to each point in the sample space $s_i$ where $i = 1,2,..,N$. The following must always be true
\begin{align} \sum_{i=1}^{N} p_i = 1 \end{align}
In other words, the sum of the probability for each point in our sample space $S$ has to equal to 1.

We can further idea as follows: define for any event $\mathcal{E}$
\begin{align*}
  P(\mathcal{E}) = \sum_{i:s_i \in \mathcal{E}} p_i
\end{align*}
In words, the equation states that the probability of an event $\mathcal{E}$ is the sum of the probabilities that belong to the points in the sample space which coincide with $\mathcal{E}$. This concept is very simple but the equation above looks very daunting, an example should help.

\de{
  \b{Example 2.16} There is a sample space $S$ with the sample points \{$a,b,c,d$\} with respected probabilities \{$0.1,0.3,0.5,0.1$\}. Let there be an event $\mathcal{A}$ = \{$a,b$\}, and let there be an event $\mathcal{B}$ = \{$b,c,d$\}. Find the probability of $\mathcal{A}$ and the probability of $\mathcal{B}$.
  \\
  
  \b{Ans.} $P(\mathcal{A}) = 0.1 + 0.3 = 0.4$; $P(\mathcal{B}) = 0.3 + 0.5 + 0.1 = 0.9$.

  Notice the sum of the probability of all events in the sample space equals to 1.
}

There is a special case which considers equally likely outcomes corresponds to $p_i = \frac{1}{N}$, where $N$ is the total number of outcomes. If we look at the event $\mathcal{E}$ then it's always the total number of outcomes which produce $\mathcal{E}$ over the total number of outcomes $N$.
\begin{align}
  P(\mathcal{E}) = \frac{|\mathcal{E}|}{N}
\end{align}

\de{
  \b{Example 2.18} There are 50 components, containing 47 working and 3 non-working. If we sample 6 components, what is the probability of getting exactly 2 defectives

  \b{Ans.} The total number of outcomes
  \begin{align*} N = {50 \choose 6} = 15890700 \end{align*}
  The number of sample points in our sample space in $\mathcal{E}$
  \begin{align*} |\mathcal{E}| = {3 \choose 2} {47 \choose 4} = 535095 \end{align*}
  Thus the probability of $\mathcal{E}$
  \begin{align*} P(\mathcal{E}) = \frac{|\mathcal{E}|}{N} = \frac{535095}{15890700} = 0.034 \end{align*}  
}

\newpage

If we decided to repeat the calculations of Example 2.18 with $X$ number of defective parts in our sample of 6 components, we get the following distribution

\begin{center}
\begin{tabular}{cc}
  $X$ & $P[X]$ \\ \toprule
  0 & 0.676 \\
  1 & 0.290 \\
  2 & 0.034 \\
  3 & 0.010
\end{tabular}
\end{center}

\de{
  \b{Exercise 2.66} Credit cards contain 16 digits. But only $10^6$ digits are valid. If you were to pick a 16 digit number at random, what is the probability it is a working credit card number.

  \b{Ans.} The total number of credit card number can be calculate by utilizing the rule of multiplicity
  \begin{align*} N = 10^6 \end{align*}
  The number of card numbers of 16 digits which actually work is
  \begin{align*} |\mathcal{E}|  = 10^8 \end{align*}
  Thus the probability of choosing a valid credit card number is
  \begin{align*} P(\mathcal{E}) = \frac{\mathcal{|E|}}{N} = \frac{10^8}{10^{16}} \end{align*}
}

\de{
  \b{Exercise 2.68} A message can follow different server paths on a network to get to the same destination. The first server stack consists of 5 servers, the second server stack contains 5 servers, and the third server stack has 4 servers.

  \begin{enumerate}
    \item How many paths are available for a message to take?

    \b{Ans.} \begin{align*}N = 5 \cross 5 \cross 4 = 100 \end{align*}
    \item What is the probability the message will go through the first server on the third server stack?

    \b{Ans.} \begin{align*}P(\mathcal{E}) = \frac{\mathcal{|E|}}{N} = \frac{5 \cross 5 \cross 1}{100} = \frac{1}{4}\end{align*}
    \end{enumerate}  
  }

  \subsection{Axioms of Probability}

  A probability $P$ is assigned to each subset of outcomes (event $\mathcal{E}$) in a collection of subset of outcomes (the sample space $S$) of a random experiment. $P$ has to satisfy the following assumptions:

  \begin{enumerate}
    \item $P(S) = 1$ : The probability for all events in the sample space must be equal to 1
    \item $0 \leq P(\mathcal{E}) \leq 1$ : The probability for any subset of events in the sample space must be greater than zero and less than or equal to 1
    \item $E_1 \cap E_2 = \emptyset \implies P(E_1 \cup E_2) = P(E_1) + P(E_2)$ : If two events are \b{mutually disjoint} (they happen at the same time, but do not influence the other), then the probability of one event or the other occurring is the sum of their individual probabilities
  \end{enumerate}

  By using the result of axiom 3, we are able to extend it to $n$ events.

  If $E_i \cap E_j = \emptyset $ for all $i \neq j$ then
  \begin{align}
    P(\cup_{i=1}^{n} E_i) = \sum_{i=1}^{n} P(E_i)
  \end{align}
  If the number of potential events is infinite, then
    \begin{align}
    P(\cup_{i=1}^{\infty} E_i) = \sum_{i=1}^{\infty} P(E_i)
    \end{align}

    If $A$ and $B$ are not mutually exclusive, then we can draw Venn diagrams to show
    \begin{align}
      P(A\cup B) = P(A) + P(B) - P(A \cap B)
    \end{align}
    The reason we subtract $P(A \cap B)$ is because we counted that specific area twice when adding the probability of $A$ and $B$

    Likewise, for events $A, B, C$
    \begin{align}
      P(A \cup B \cup C) = P(A) + P(B) + P(C) - P(A \cap B) - P(A \cap C) - P(B \cap C) + P(A \cap B \cap C) 
    \end{align}
    You can prove this using visual proofs or using a direct proof by elements.

    \de{
      \b{Exercise 2.82} $P(A) = 0.3$, $P(B) = 0.2 $, $ P(A \cap B) = 0.1$ : what is $P(A\cup B)$? \vspace{2mm}

      \b{Ans.} $P(A \cup B) = P(A) + P(B) - P(A \cap B) = 0.2 + 0.3 - 0.1 = 0.4$
    }

    \de{
      \b{Exercise 2.95} A bar code must have black bars separated by white bars. There are 3 narrow black bars denoted by $b$, and two wide black bars denoted by $B$. Likewise, there are 3 narrow white spaces denoted by $w$, and only 1 wide white space denoted by $W$(A bar considers only the black, a space considers only the white).\vspace{2mm}

      \begin{enumerate}
        \item What is the probability the first bar is wide or the second bar is wide? \vspace{2mm}
        
        \b{Ans.} Let $A$ be the event the first bar is wide, and let $B$ be the event the second bar is wide.
        \begin{align*}
          P(A \cup B) &= P(A) + P(B) - P(A \cap B) \\
          &= \dfrac{\dfrac{4!}{1!3!} + \dfrac{4!}{1!3!} - \dfrac{3!}{3!}}{\dfrac{5!}{2!3!}} = 0.7
        \end{align*}

        \item What is the probability the first nor the second bar is wide? \vspace{2mm}

        \b{Ans.}
        \begin{align*}
          P(A'\cap B') = \frac{\dfrac{3!}{1!2!}}{\dfrac{5!}{2!3!}} = \frac{3}{10} = 0.3
        \end{align*}
        The easier calculation is simply the complement
        \begin{align*}
          P(A' \cap B') = 1 - P((A' \cup B')') = 1- P(A \cap B) = 1-0.7 = 0.3
        \end{align*}        
        \end{enumerate}
      }
      \de
      {
        \begin{enumerate}
          \setcounter{enumi}{2}
          \item What is the probability the first bar is wide or the second bar is narrow?
          \vspace{2mm}

          \b{Ans.} Let $A$ be the event the first bar is wide, and let $B$ be the event the second bar is narrow.
          \begin{align*}
            P(A \cup B)  = P(A) + P(B) - P(A \cap B) = \dfrac{\dfrac{4!}{1!3!} + \dfrac{4!}{2!2!} - \dfrac{3!}{1!2!}}{\dfrac{5!}{2!3!}} = 0.7
            \end{align*}
          \end{enumerate}
        }

        \de{
          \b{Example 2.97} Passwords are exactly 8 characters wrong with repetitions allowed from 26 lower case, 26 upper case, and 10 integers. If $A =$ \{only letters\} and $B = $\{ only integers\} Find:

          \begin{enumerate}
            \item $P(A \cup B)$
            \vspace{2mm}
            
            \b{Ans.} \begin{align*}
                       P(A \cup B) = P(A) + P(B) - P(A \cap B) = \frac{52^8}{62^8} + \frac{10^8}{62^8} - 0 
                       \end{align*}
           \item $P(A' \cup B)$
           \vspace{2mm}
           
           \b{Ans.} \begin{align*}
                      P(A' \cup B) = \frac{10^8}{62^8} 
                    \end{align*}
           \item $P($exactly 1 or two integers$)$
           \vspace{2mm}

           \b{Ans. } Let $A$ be the event of exactly 1 integer, and let B be the event of exactly 2 integers
           \begin{align*}
             P(A) &= \dfrac{{ 8 \choose 1 }  \cross 10 \cross 52^7}{62^8}\\[2mm]
             P(B) &= \dfrac{{ 8 \choose 2 } \cross 10^2 \cross 52^6}{62^8}\\[2mm]
             P(A \cup B) &= P(A) + P(B) = 0.6302
           \end{align*}
           
            \end{enumerate}
          
        }
    
  \end{document}