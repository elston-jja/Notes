\documentclass[12pt, titlepage, oneside]{article}

\usepackage[margin=0.5in]{geometry}

\usepackage{siunitx, booktabs, amsmath, enumitem, pdfpages,mathrsfs,tabularx,caption, graphicx, pgfplots, textcomp,wrapfig, commath, svg}

\usepackage{amssymb}

\usepackage{parskip}

\usepackage[siunitx]{circuitikz}
\sisetup{detect-weight=true, detect-family=true}

\setlength\parindent{0pt}

\let\oldhat\hat
\let\oldvec\vec
\newcommand{\cross}{\bm{\times}}
\renewcommand{\hat}[1]{\oldhat{\mathbf{#1}}}

\usepackage{bm}
\renewcommand{\vec}[1]{\oldvec{\bm{#1}}}
\renewcommand{\hat}[1]{\oldhat{\bm{#1}}}
\renewcommand{\b}[1]{\textbf{#1}}

\newcommand{\de}[1]{\noindent\fbox{\parbox{\textwidth}{#1}}}

\newcommand{\be}{\begin{equation*}}
\newcommand{\ee}{\end{equation*}}

\begin{document}
	\setcounter{section}{9}
	\setcounter{page}{26}
    \section{Normal Distribution Approximations}
    \subsection{Normal Approximation to Binomial}
    \begin{itemize}
      \item Normal arises when adding many similar independent quantities.
      \item Binomial is the sum of independent binomial random variables
    \end{itemize}
    Let $X \thicksim Binomial(n,p)$. We know $\mu = np$, and $\sigma^2 = np(1-p)$. Normal approximation says that $P(X \leq x)$ can be approximated by a normal
    \begin{align}
      X \thicksim N(np,np(1-p))
    \end{align}
    And can be evaluated using normal tables by standardization if $n$ is sufficiently large
    \begin{align}
      P(X \leq x) = P\Bigg(Z \leq \frac{x - np}{\sqrt{np(1-p)}}\Bigg)
    \end{align}
    We need to make sure that the $\mu = np$ and $\sigma^2 = np(1-p)$ is greater than 5 ($\mu, \sigma^2 > 5$)

    The reason for the mean and the variance to be greater than 5, is because the approximation only works when the distribution is near the center.

    \de{
      \b{Example 4.17} Digital communication channel receives $X$ number of bits as error and can be modelled as a $Binomial(16000000, 0.00001)$.
      \\

      \b{Ans.} The answer to this question is very difficult to compute
      \begin{align*}
        P(X \leq 150) = \sum_{x = 0}^{150} \binom{16000000}{x}(0.00001)^x(0.99999)^1{6000000}
      \end{align*}
    }

    There is one additional note about the normal approximation to the binomial distribution: we have to get a bit more accurate. We add and subtract 0.5 from $x$ in order to account for a wider distribution.
    \begin{align}
      P(X \leq x) = P(X \leq x + 0.5 ) = P\Bigg(Z \leq \frac{x+0.5 - np}{\sqrt{np(1-p)}}\Bigg)\\[2mm]
      P(x  \leq X) = P( x - 0.5 \leq X ) = P\Bigg(\frac{x-0.5 - np}{\sqrt{np(1-p)}} \leq Z\Bigg)
    \end{align}

    \de{
      \b{Example 4.18} Recall Example 4.17 above. We can compute that much more easily with a normal distribution approximation

      We know $\mu = np = 160$, $\sigma = \sqrt{np(1-p)} = 12.6490$, thus
      \begin{align*}
        P(X \leq 150) = P\bigg(Z \leq \frac{150-160}{12.6490}\bigg) = P(Z \leq -0.7906) = 0.2146
      \end{align*}
      With correction
      \begin{align*}
        P(X \leq 150) = P\bigg(\frac{150+0.5 - 160}{12.6490}\bigg) = P(Z \leq -0.7510) = 0.2263
        \end{align*}
 The exact answer computed by Matlab is 0.288, so we see that the correction is a better result.     
      
      
}
\subsection{Normal Approximation to Hypergeometric}
Recall that a binomial distribution approximates a Hypergeometric when the sample size $n$ is much smaller than the population size $N$. Text suggests $\frac{n}{N} < 0.1$. More conservative range requires $\frac{n}{N} < 0.05$. Here $p=\frac{K}{N}$. So here $p = \frac{K}{N}$. So if $\frac{n}{N} < 0.05$ and $n$ is large, we can use the normal approximation to the Hypergeometric. Look at examples on the next page.
  \end{document}
