\documentclass[12pt, titlepage, oneside]{article}

\usepackage[margin=0.5in]{geometry}

\usepackage{siunitx, booktabs, amsmath, enumitem, pdfpages,mathrsfs,tabularx,caption, graphicx, pgfplots, textcomp,wrapfig, commath, svg}

\usepackage{parskip}

\usepackage[siunitx]{circuitikz}
\sisetup{detect-weight=true, detect-family=true}

\setlength\parindent{0pt}

\let\oldhat\hat
\let\oldvec\vec
\newcommand{\cross}{\bm{\times}}
\renewcommand{\hat}[1]{\oldhat{\mathbf{#1}}}

\usepackage{bm}
\renewcommand{\vec}[1]{\oldvec{\bm{#1}}}
\renewcommand{\hat}[1]{\oldhat{\bm{#1}}}
\renewcommand{\b}[1]{\textbf{#1}}

\newcommand{\de}[1]{\noindent\fbox{\parbox{\textwidth}{#1}}}

\newcommand{\be}{\begin{equation*}}
\newcommand{\ee}{\end{equation*}}

\begin{document}

\setcounter{section}{1}
\setcounter{page}{3}

\section{Counting Techniques: For Counting Problems}

\subsection{Multiplicity}

Suppose we are required to perform a task $n$ which has $k$ steps, but each step does not affect the step before or the step after ($n_{k-1}$ and $n_{k+1}$ are not affected). There is a rule known as the \b{Multiplicity} which states that the sample space would be
\begin{align}
|S| = n_1 \cross n_2 \cross n_3 \cross ... \cross n_k
\end{align}
Think about flipping a coin 3 times: there are 2 possible outcomes for each coin toss, and the previous toss does not affect the toss after. Since there would be two possible outcomes $3$ times, our sample space would be $2 \cross 2 \cross 2 = 2^3 = 8$

\subsection{r - Permutations}

Suppose we are required to know how many different ways to organize $n$ \b{distinct} objects in $r$ spots, we realize this is a \b{Permutation} problem.

We have 2 cases for permutations:

If we have $n$ spots and $n$ objects ($r = n$): each time we place an object, the next object has one fewer spots to occupy, so by the multiplicity rule we get $n!$
\begin{align*}
  n! = n \cross (n-1) \cross (n-2) \cross ... \cross 2 \cross 1 
  \end{align*}
  If we have $n$ spots and $r$ objects ($r < n$): each time we place an object, the next object has one fewer spots to occupy, so the multiplicity rule states we get $n!/(n-r)!$
  \begin{align*}
    \frac{n!}{(n-r)!} = \frac{n \cross (n-1) \cross (n-2) \cross ... \cross (n-r+2) \cross (n-r+1) \cross (n-r)!}{(n-r)!}
  \end{align*}
    the $(n-r)!$ cancel out
  \begin{align*}
    \frac{n!}{(n-r)!} = n\cross (n-1) \cross (n-2) \cross ... \cross (n-r+2) \cross (n-r+1)
  \end{align*}
  The formula above is known as the general formula of a permutation, the notation is written as nPr.
  \begin{align}
    _n\text{P}_r = \frac{n!}{(n-r)!} = n\cross (n-1) \cross (n-2) \cross ... \cross (n-r+2) \cross (n-r+1)
  \end{align}
  \de{
    \b{Example 2.10} Imagine we have a printed circuit board with 8 locations for 4 \b{distinct} components. How many ways can we place the 4 distinct chips?
    
    \b{Ans.} Since we know that each component is distinct we can solve this using the multiplicity rule. The first time we place a chip there are 8 places, then for the next chip we have 7 places, then 6 places for the next, and so on.
    \begin{align*}
      8 \cross 7 \cross 6 \cross 5 = 1680
    \end{align*}
    We can also use r-permutations to solve this problem faster.
    \begin{align*}
      _8\text{P}_4 = \frac{8!}{(8-4)!} = \frac{8\cross 7 \cross 6 \cross 5 \cross 4!}{4!} = 8\cross 7 \cross 6 \cross 5
      \end{align*}   
    }

    \subsubsection{Permutations of Similar Items}

    Suppose we have $n$ spots and we need to place $n$ objects. The difference here is our $n$ objects are composed of sets of similar type objects. That is $n_1$ is the set of one type of objects, $n_2$ is the set of another type, so on and so forth.

    A more mathematical approach: $n_i$ sets of objects of type $i$, where $i = 1,2,..,r$ and $n = n_1 + n_2 + ... + n_i$. The number of distinct permutations of $n$ in $n$ spots is
    \begin{align}
      \frac{n!}{n_1! n_2! ... n_r!}
    \end{align}
    \b{Proof:} Suppose all objects were distinct, thus we would have $n!$ distinct ways of organizing such objects. Now if we consider the identical types, we see that for each type $i$ we counted by $n_i$ times over. Thus we divide by their factorial to eliminate the identical combinations.
    
    \de{
      \b{Example 2.11} In a hospital there are 3 knee surgeries and 2 hip surgeries on the schedule for a surgeon. How many ways can the surgeon organize their schedule?

      \b{Ans.} The total combinations of surgeries which can occur is $5!$, but since $3!$ combinations considers identical knee surgeries, and $2!$ combinations considers identical hip surgeries. We have to divide to remove the combinations that are not identical.
      \begin{align*}
        \frac{5!}{3!2!} = 10
      \end{align*}
      Thus there are 10 different ways we can organize 3 knee surgeries and 2 hip surgeries. 
    }

    \subsubsection{r-Permutations of Similar Items}

    Suppose we have $n$ spots and $r$ objects ($r < n$). $r$ contains sets of identical objects of type $i = 1,2,..,k$, such that $r = r_1 + r_2 + .. + r_k$. As always, we consider the distinct case and divide by the sets of identical combinations: if there were $n$ spots and $r$ distinct objects then we have $_n\text{P}_r$ distinct combinations. Now we divide by the sets of identical combinations,
    \begin{align*}
      \frac{_n\text{P}_r}{r_1!r_2!...r_k!} = \frac{n!}{(n-r)! \cross (r_1!r_2!...r_k!)}
    \end{align*}

    \de{
      \b{Example: } Suppose we have 8 spots on a printed circuit board and two sets of 2 identical chips, how many distinct combinations do we have?

      \b{Ans.} The distinct case would be $_8\text{P}_4$ different combinations, but now since there are 2 identical chips of 1 type and 2 identical chips of another type, we divide by $2! 2!$.
      \begin{align*}
        \frac{_8\text{P}_4}{4!} = \frac{8 \cross 7 \cross 6 \cross 5}{2! 2!} = 420
        \end{align*}      
      }
      \subsection{r-Combinations}

      Suppose we have $n$ \b{distinct} items and we want to pick a set of $r$ items. In other words, we have $n$ distinct items and we want to choose a set of $r$ items. We see that each time we pick an item, we have a smaller set to choose from so it resembles $_n\text{P}_r$, but remember we are picking sets (not individual items), so we must divide by the total amount of redundant sets which is $r!$. The notation we use is $_n\text{C}_r$
      \begin{align}
        _n\text{C}_r = {n \choose r} = \frac{_n\text{P}_r}{r!} = \frac{n!}{r!(n-r)!}
      \end{align}

      \de{
        \b{Example 2.13} Suppose we have a printed circuit board with 8 locations for 5 identical components. How many ways can we design this board?

        \b{Ans.} This is the same as performing an r-Permutation with 1 set of identical components.
        \begin{align*}
          \frac{_8\text{P}_5}{5!} = {8 \choose 5} = \frac{8!}{(8-5)! \cross 5!} = 56
          \end{align*}
        }

        \de{
          \b{Example 2.14} Suppose you buy 6 parts from a bin of 50 without replacement. In the bin of 50, 3 parts are defective and 47 parts are up to specifications. How many different combinations of the 6 parts you buy contains 2 defectives?

          \b{Ans.} Since we have 3 parts defective, we need to find the total combinations of being able to pick 2 defective parts: $_3\text{C}_2$. Now we need to pick the rest of the parts from the working parts, so we get $_{47}\text{C}_4$. By multiplicity, we can multiply these steps together to get the total combinations involving 2 defective parts.
          \begin{align*}
            {3 \choose 2}{47 \choose 4} = 535095
            \end{align*} 
          }

          \de{
            \b{Example 2.41} Suppose we have produced 140 chips and know that 10 chips produced are not up to specification. An officer performs an inspection of 5 chips. How many samples can the officer find at least 1 chip not up to specification.

            \b{Ans.} Since we are looking for at least one chip not up to specification, we must also consider the combinations where there are more than 1 chip which does not match specification. Like the last example, we first choose the amount of chips not up to specification, then choose working chips for the rest. 
            \begin{align*}
              {10 \choose 1} {130 \choose 4} +
              {10 \choose 2} {130 \choose 3} +
              {10 \choose 3} {130 \choose 2} +
              {10 \choose 4} {130 \choose 1} +
              {10 \choose 5} {130 \choose 0}
              \end{align*}
              This method is tedious, and there is a faster method. We can take the complement where we find the total combinations which can be chosen then subtract the combinations where all the chips are working up to specification.
              \begin{align*}
                {140 \choose 5} - {130 \choose 5}{10 \choose 0}
              \end{align*}

              Both methods yield the same result, but one is much more faster and can be useful when working on exam problems.
          }

\end{document}