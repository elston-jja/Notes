\documentclass[12pt, titlepage, oneside]{article}

\usepackage[margin=0.5in]{geometry}

\usepackage{siunitx, booktabs, amsmath, enumitem, pdfpages,mathrsfs,tabularx,caption, graphicx, pgfplots, textcomp,wrapfig, commath, svg}

\usepackage{parskip}

\usepackage[siunitx]{circuitikz}
\sisetup{detect-weight=true, detect-family=true}

\setlength\parindent{0pt}

\let\oldhat\hat
\let\oldvec\vec
\newcommand{\cross}{\bm{\times}}
\renewcommand{\hat}[1]{\oldhat{\mathbf{#1}}}

\usepackage{bm}
\renewcommand{\vec}[1]{\oldvec{\bm{#1}}}
\renewcommand{\hat}[1]{\oldhat{\bm{#1}}}
\renewcommand{\b}[1]{\textbf{#1}}

\newcommand{\de}[1]{\noindent\fbox{\parbox{\textwidth}{#1}}}

\newcommand{\be}{\begin{equation*}}
\newcommand{\ee}{\end{equation*}}

\begin{document}
	
	\setcounter{section}{4}
	\setcounter{page}{13}
	
	\section{Independence}
	
	When two probabilities are \b{independent} it means that knowledge that other occurred, does not change the probability of the other. Mathematically speaking, we arrive at the conclusion below
	\begin{align}
	P(B|A) = P(B)
	\end{align}
	From this equation we can see that the occurrence of event $A$ does not change $P(B)$. Since the probabilities do not affect each other, we call $A$ and $B$ independent events. We can recall from the earlier section
	\begin{align}
	P(B|A) = \frac{P(B \cap A)}{P(A)} 
	\end{align}
	In order to obtain equation (1) from equation (2), the following has to be true 
	\begin{align}
	P(A \cap B) = P(A) P(B)
	\end{align}
	This answer follows symmetry as $P(A \cap B) \equiv P(B \cap A)$.

    The following are of independent events: $A', B$ or $A', B'$ or $A, B'$. For each pair we see that the probability of the other does not get affected. For example, $P(A'|B)$ since we know that $B \subset A'$ so it does not change the probability of $A'$ occurring.
	
	Events $A_1, A_2, ... , A_n$ are independent if
    \begin{align}
      P(\cap^k_{j=1}A_{i_j}) = \prod^k_{j=1} P(A_{i_j})
      \end{align}
      for any $k \geq 2$ and any subset of events \{ $ A_{i_1}, ... , A_{i_k} $ \}

      Equation (4) simply states that any event $A_n$ is independent if the intersection of any subset\b{s} $A_{i_k}$, where $k \geq 2$ would simply yield the product of the probability of the events.

      \de{
        \b{Example } 30\% of a company's washing machines require services under warranty, 10\% of its dryers needs service as well. If someone pays for a washer and dryer, what is the probability that both machines require service? Assume independence
        \\

        \b{Ans.} Let $A$ be the event that the washer needs service, and let $B$ be the event that the dryer needs service.
        \begin{align*}
          P(A \cap B) = P(A)P(B) = (0.1)(0.3) = 0.03
          \end{align*}
        }

        \subsection{Reliability}
        The reliability of a device or system is the probability that it operates for a specified duration.

        Let $A$ and $B$ be components in a circuit

        For parallel circuits, the reliability would be the probability of either component working (this is because if one or the other works, the circuit will work).
        \begin{align*}
          P(\text{Reliability}) = P(A \cup B) = P(A) + P(B) -P(A \cap B)
          \end{align*}
          For series circuits, the reliability would be the probability of both components working (this is because if one or the other fails, the circuit will not work).
          \begin{align*}
            P(\text{Reliability}) = P(A \cap B) = P(A) P(B)
          \end{align*}
          If given probability of failure and asked for the total failure probability, then parallel would be the probability of both failing, and series would be the probability of either failing.
\end{document}
